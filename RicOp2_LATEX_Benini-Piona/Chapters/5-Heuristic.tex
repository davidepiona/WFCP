% !TEX TS-program = pdflatex
% !TEX root = ../ArsClassica.tex

%************************************************
\chapter{Heuristic Methods}
\label{chp:5-Heuristic}
%************************************************
As we have said, the WFCP belongs to the class of NP-Hard problems so, many times, when the number of nodes is too high, we can’t obtain an optimal solution in a feasible time.
For this reason we have decided to implement some algorithms that, instead of solving the mathematical model, use heuristic methods to find a solution of the problem. These methods can compute, relying on the characteristic structure of the problem, a good solution quickly but that is not guaranteed to be optimal.
Then, in the following section, we are going to describe some heuristic methods that iteratively execute a specific procedure trying to obtain consecutively a better solution.
\section{Dijkstra}
Dijkstra's algorithm is an algorithm for finding the shortest paths between nodes in a graph, it was conceived by computer scientist Edsger W. Dijkstra in 1956 and published three years later. We use the variant that fixes a single node as the "source" node and finds the shortest path from the source node to all other nodes in the graph producing the shortest-path tree. \\
(come funziona l'algoritmo???)
The resulting graph did not respect the $C$ constraint and can have crossing cables. 
\section{GRASP}
The GRASP algorithm, that means \textit{Greedy Randomized Adaptive Search Procedures}, is a metaheuristic algorithm first introduced by Feo and Resende (1989). GRASP typically consists of iterations made up from successive constructions of a greedy randomized solution and subsequent iterative improvements of it through a local search. \\
This algorithm, that we implemented for the Wind Farm Cable Problem, consists in looking at the best $N$ possible choices for each iteration, then flip a coin and if it comes out head choose the best one, otherwise pick one solution randomly. 
[è giusto il funzionamento ??? spiegare altro ... anche su pdf amici] 
\section{1-Opt}
It belongs to the cathegory \textit{Refining Heuristic Algorithm}, therefore it tryes to                                                                  improve an already existing solution.                                                                   \\
Basically it tryes to substitute an arc with another one that reduces the cost of the objective function. \\
\section{Multistart}

\section{Taboo Search}

\section{Ant Algorithm}
[??? questo algoritmo è esattamente uguale a quello nel pdf .. credo che vada cambiato un po']
From the \cite{nedlin2017ant} research: 
\begin{algorithm}[H]
\caption{: FindPath in Kruskal Approach} \label{alg:SC}
\begin{algorithmic} 
\STATE{\textbf{Input:} Graph $G = (E, V )$, Pheromone Values $P_E$}
\STATE{\textbf{Data:} Current Node $c$, Choice List $L$}
\STATE{\textbf{Output:} $T \subseteq E$}
\REPEAT
\STATE {\textsc{L.CLEAR()}}
\STATE {// select all valid neighbors}
\STATE {\textbf{forall} $(c,v) \in E \smallsetminus \{ e \in E \ | \ T \bigcup \ \{ e \}$ \textit{ contains a circle} \} \textbf{do}} 
\STATE {\quad \textbf{if} \textit{c or v are not connected to a substation in ($V,T$)} \textbf{then}}
\STATE {\quad \quad \textsc{L.INSERT}\textit{(c, v)}}
\STATE {\textbf{if} $L$ \textit{is not empty} \textbf{then}}
\STATE {\quad Select a random edge e = \{ $c,v$ \} from $L$ using $P_E$ as weight}
\STATE {\quad $T \leftarrow T \ \bigcup \ \{ e \}$}
\UNTIL {$L$ \textit{is empty}}
\end{algorithmic}
\end{algorithm}
\section{Results}



