% !TEX TS-program = pdflatex
% !TEX root = ../ArsClassica.tex

%************************************************
\chapter{Matheuristic Methods}
\label{chp:4-Matheuristics}
%************************************************
Given that the \textsc{CPLEX} solver is really optimized, we apply the heuristic in the model writing. In this way the model that we’ll give to \textsc{CPLEX} should be theoretically easier to solve.

\section{Hard Fixing}

\subsection{Random Hard Fixing}

\subsection{RINS Hard Fixing}



\section{Soft Fixing}
This method, also called \textit{local branching}, given a solution called $y^{REF}$, fixes at least a percentual of the arcs of that solution, and repeat the execution searching the best choice for the others. A critical issue of variable fixing methods is related on the choice of the variables to be fixed at each step and wrong choices are typically difficult to detect. In this sense, the purpose of the soft fixing is to fix a relevant number of variables without losing the possibility of finding good feasible solutions.\\
A possible implementation is, give the solution $y^{REF}$ represented by an array of zeros and ones, given a generic solution $y$ and given a constant $K$:
\[
	y^{REF} = (0,1,0,1,1,...)
\]
\[
	\sum_{(i,j):y^{REF}_{ij}=0} y_{ij} + \sum_{(i,j):y^{REF}_{ij}=1} (1-y_{ij}) \quad \leq K
\]
It represents the Hamming distance between $y$ and $y^{REF}$; in practice the constraints allows one to replace at most $K$ edges of $y^{REF}$. \\
Then, in our implementation, the algorithm starts producing a heuristic solution $Y$, adds the local branching constraint to the MIP model and solves it using \textsc{CPLEX}.\\
[come abbiamo scelto K ???]\\
Soft fixing avoids a too rigid fixing of the variables in favor of a more flexible condition; this allows the new solution to "move" from the older one fixing at each iteration some random arcs and moving the other looking quickly for a better solution. (???) \\
This is the symmetric version of this method because it considers equally the 0-1 and the 1-0 flips. 
[inserire immagine grafico local brancing???]
\subsection{Asymmetric Soft Fixing}
In this case we consider only the flips from 1 to 0:
\[
	\sum_{(i,j):y^{REF}_{ij}=1} (1-y_{ij}) \quad \leq K
\]
\[
	\sum_{(i,j):y^{REF}_{ij}=1} y_{ij} \quad \geq \sum_{(i,j):y^{REF}_{ij}=1} 1 - K
\]
\[
	\sum_{(i,j):y^{REF}_{ij}=1} y_{ij} \quad \geq n - 1 - K
\]
This method is more convenient from the graphical point of view. 
[??? aggiungere .. bo qualcosa sull'integrality grip?]
\subsection{Symmetric RINS}

\subsection{Asymmetric RINS}

\section{Results}


